
% Thesis Abstract -----------------------------------------------------


%\begin{abstractslong}    %uncommenting this line, gives a different abstract heading
\begin{abstracts}        %this creates the heading for the abstract page

The proliferation of mobile devices over the past several years has created a whole new world of the Internet. The deluge of applications for every aspect of today�s life has raised the expectation of having ubiquitous connectivity, with a desired Quality of Service (QoS).  Although appealing, it has violated the original Internet design which was not intended to support mobility, neither better than best-effort delivery.

    It is also a well-known fact that technology is an ever-advancing need of the human society, and undeniably the Internet forms a major part of our lives now. Everyday more and more users flood the Internet with enormous amount of data and information. As such there is a need to effectively handle all the information and traffic in a way that there is an availability of high speed network routing without any loss in data transmission. 

     QoS provisioning has been one of the long lasting focuses in the network research community. While designed for fixed networks, the use of QoS protocols in IP-based mobile networks, where hosts dynamically change their point of attachments, imposes new challenges to be studied and analysed. Furthermore, a massive growth in the access network traffic with its highly unpredictable nature can cause bottlenecks in some links while others are under-utilised, rendering the load skewed, and therefore, breaching the QoS provisioning commitments. 

     The main objective of this research is to propose a novel QoS mechanism for mobile networks. The new scheme is composed of two different approached accountable for QoS provisioning in next-generation access networks. Firstly, a new method is proposed that minimises the signalling overhead, as well as the interruption in QoS at the time of handover. Through a developed analytical framework and simulation scenario, the performance of the new scheme is investigated thoroughly, with the focus on the figures of merit that affect the efficiency of using QoS signalling protocols in access networks. Secondly, a new QoS-aware routing mechanism is proposed, based on the OSPF protocol, intending to minimise the congestion on the links while at the same time complying with traffic requirements. OSPF was created for providing flexibility and great scalability, and although not widely used shows strong promises for the future. 

     This research delves into the study and development of IP-based networking, built upon an extension to OSPF routing protocol, that will foster integrated functioning of technologies that currently lead the vision for the novel telecommunication infrastructures and service provision. This novel QoS-aware approach, Multi-Plane Routing (MPR), is applied in access networks for IP routing. MPR divides the physical network into several logical routing planes, each being associated with a dedicated link weight configuration. Network topology and node degree distribution directly impact the performance of our strategy. 

The foundation of the project�s vision for networking in the future networks is in the evolution and derivatives of IP routing that are inherited from the native Internet and stand as the solution for networking in the sought �all-IP� integrated modern telecommunications infrastructures. MPR uses QoS-awareness and policies for plane selection to achieve optimal performance according to criteria selected for traffic engineering in networks. 

\end{abstracts}
%\end{abstractlongs}


% ----------------------------------------------------------------------


%%% Local Variables: 
%%% mode: latex
%%% TeX-master: "../thesis"
%%% End: 
